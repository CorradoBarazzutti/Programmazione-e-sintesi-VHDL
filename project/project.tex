\chapter{Implementazione}

La descrizione in linguaggio VHDL del componente è comportamentale e si avvale di un processo che garantisce la sequenzialità dell'esecuzione.

Per programmare il componente è stato adottato un approccio top-down che si rispecchia nella struttura di questo capitolo. In primo luogo è definita la struttura di controllo principale e in seguito le procedure che realizzano le funzionalità necessarie.

\section{Struttura di controllo}
La struttura di controllo garantisce la corretta evoluzione della rete, che può essere quindi vista come una macchina a stati, insieme alla sincronizzazione con i segnali di $reset$, $start$ e $clock$. Questo si implementa inserendo i segnali nella sensitivity list del processo, quindi con una strutura if-then-else. Il processo attende in primo luogo che il segnale START in ingresso verrà portato a
1 per un ciclo di clock. In seguito a questo evento il modulo non registrà ne un nuovo segnale di start 
ne uno di $reset$ fino alla fine della elaborazione. 
Al termine della eleborazione e una volta scritto il risultato in memoria, il modulo porta a 1 il segnale di DONE  per 1 ciclo di clock. \\

L'elaborazione va immaginata come divisa in tre fasi: iniziaizazzione, induzione e scrittura dei risultati. Come suggerito da \cite{brandolese} nel capitolo sulle macchine a stati l'evoluzione è controllata da una struttura if-then-else. La condizione di evoluzione della macchina è data da un contatore: la prima fase dura 3 cicli di clock, cioé il tempo di leggere l'intestazione, la fase di induzione dura $ROWS \times COLS$ cicli di clock, la scrittura dei risultati 2 cicli.\\

\section{Inizializzazione}
\label{inizializzazione}
La fase di inizializzazione consiste nella lettura in memoria dell'intestazione che contiene la taglia dell'immagine (ROWS e COLS) e il colore soglia che definisce il limite della figura di interesse (TRIG).
Grazie a queste informazioni sono quindi inizializzati, al loro valore pessimo, i parametri di ottimizzazione. Il limite sinistro è inizializzato al bordo destro, cioè al numero di colonne dell'immagine mentre il limite destro è inizializzato a zero. In modo analogo sono inizializzato $bottom$ e $top$. 

\section{Ricerca della soluzione}
\label{induzione}
Il calcolo dell’area che circoscrive la figura di interesse si riconduce a trovare il valore della variabili $left$, $reight$, $bottom$, $top$, cioè i limiti sinistro, destro dell'immagine espressi come indici di riga e i limiti superiore e inferiore come indici di colonna. \\

Tale ricerca consiste in un processo di ottimizzazione sull'immagine in memoria. Per ottimizzazione si intende quella delle variabili left, reight, bottom, top seconda la condizione trattata nella sezione \ref{update}.
Notare che lo spazio di ricerca è di taglia limitata $2^8 \times 2^8$ e può quindi essere esplorato con un ciclo $for$ dove l'estremo è definito da una constante. Nella nostro impementaione il ciclo è realizzato con il contatore sopracitato. Il contatore a 9 bit puo contare fino a 512 ma è resettato a 3 nella fase di inizializzazione, a $ROWS \times COLS$ in induzione e a 2 in scrittura dei risultati\\

Passiamo ora a descrive nel dettagli le funzionalità.



\subsection{Aggiornamento}
\label{update}
Ad ogni cella di interesse, tale che il colore sia più scuro di quello dichiarato nell'intestazione i valori $bottom$ e $right$ sono aggiornati se più grandi dell'indirizzo di riga e di colonna della cella rispettivamente. I valori $top$ e $left$ sono invece aggiornati se inferiori alle coordinate della cella in questione.

\section {Altre utilità}

\subsection{Lettura e scrittura della memoria}
Le operazione di lettura vengo effettuate ponendo il segnale $o \textunderscore adress$ al valore binario dell'indirizzo cui si desidera accedere e il segnale di $enable$ della memoria a 1. Quindi si attende che sia trascorso mezzo ciclo di clock (un fronte di salita seguito da uno di discesa). Trascorso questo tempo, se il segnale $o \textunderscore adress$ e $enable$ sono tenuti stabili, si ha la garanzia che la memoria abbia posto il segnale $i \textunderscore data$ al valore all'indirizzo di memoria indicato. Infine si pone l'enable a zero e si può leggere il segnale $i \textunderscore data$.\\
Nello stesso modo si procede per la scrittura, avendo l'accortezza di indicare in $o \textunderscore data$ il segnale da scrivere e abilitando $o \textunderscore we$. \\
L'attesa di mezzo ciclo di clock per ogni accesso in memoria fa che questa operazione sia la più costosa in termini di tempo. Considerado, come è stato fatto nel testbench, un periodo di clock di $30 ns$ possiamo stimare il tempo di esecuzione nel peggior caso pari $T \dot 2^9 = 15 \mu s$ 

\subsection{calcolo dell'area}
Il risultato è semplicemente calcolato secondo la formula \ref{compute_area} avvalendosi della funzione $mult$ definita in sezione \ref{mult}.
\begin{equation}
	mult((UNSIGNED(right) - UNSIGNED(left)),  (UNSIGNED(bottom) - UNSIGNED(top)))
    \label{compute_area}
\end{equation}
Eventualmente può verificasi che non vi sia nessuna figura di interesse. In questa circostanza $UNSIGNED(right) < UNSIGNED(left)$ e $UNSIGNED(bottom) < UNSIGNED(top)$ perché i parametri non sono mai stati aggiornati. Il circuito rileva questa condizione e invece di effettuare la moltiplicazione ritorno correttamente che l'area è zero.

\subsection{Moltiplicazione}
\label{mult}
Per semplicità la moltiplicazione è calcolata ricastando i parametri da standard ulogic vector in interi e utilizzando l'operatore $*$. Evidentemente si potrebbe senza difficoltà ottimizzare questo aspetto implementando una moltiplicazione tra standard logic vector a 8 bit. Tale raffinamento è lasciato come possibile miglioria al sistema. 


